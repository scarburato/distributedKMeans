% !TeX spellcheck = en_US
\documentclass[parskip=full]{report}

\usepackage{amsmath}
\usepackage{amsfonts}
\usepackage{listings}
%\usepackage{beramono}
\usepackage{float}
\usepackage[utf8]{inputenc}
\usepackage[T1]{fontenc}
\usepackage{xcolor}
\usepackage[a4paper, margin={3cm}]{geometry}
\usepackage{hyperref}
\usepackage{graphicx}
\usepackage{svg}
\usepackage{subcaption}
\usepackage{float}
\usepackage{pdfpages}
\usepackage{algpseudocode}
\usepackage{algorithm}

\usepackage{tikz}

\usepackage{hyphenat}
\usepackage[english]{babel}
% Carattere monospaziato di default
\renewcommand{\ttdefault}{pcr}

\tikzstyle{block} = [draw, fill=blue!20, rectangle, 
minimum height=3em, minimum width=6em]
\tikzstyle{sum} = [draw, fill=blue!20, circle, node distance=1cm]
\tikzstyle{input} = [coordinate]
\tikzstyle{output} = [coordinate]
\tikzstyle{pinstyle} = [pin edge={to-,thin,black}]

\lstset{
	% wrap long lines on new line
	postbreak=\mbox{\textcolor{red}{$\hookrightarrow$}\space},
	breaklines=true, 
	columns=fullflexible,
	% tab and fonts
	tabsize=2,
	basicstyle=\ttfamily\small,
	% theme
	numbers=left,
	rulecolor=\color{black!30},	
	% UTF8 and escape
	escapeinside={\%TEX}{\^^M},
	inputencoding=utf8,
	extendedchars=true,
	literate={á}{{\'a}}1 {à}{{\`a}}1 {é}{{\'e}}1 {è}{{\`e}}1,
}


% Title Page
\title{
	\includegraphics[width=0.333\textwidth]{assets/unipi1.png} \\
	\textsc{University of Pisa} \\
	\vspace{.5cm}
	Artificial Intelligence and Data Engineering \\
	Cloud Computing \\
	\vspace{2cm}
	{\huge \textit{K-Means} on MapReduce}
}

\author{
	The \textbf{Don Matteo} group: \\
	\vspace{.3cm} \\
	\begin{tabular}{lr}
		Dario Pagani & 585281 \\
		Ricky Marinsalda & 585094 \\
		Giulio Bello & 603078
	\end{tabular}
}

\begin{document}
\maketitle
\tableofcontents


\chapter{Algorithm to select initial centroids}

\section{Idea}

\paragraph{}
To run the \textit{K-means} algorithm is necessary to select $k$ initial centroids, they can either be selected statically by the user --- that is as an algorithm's parameter --- or stochastically from the data; in the latter case they could either be drawn randomly with uniform probability or by employing a probability function that changes the likelihood to draw an element accordingly to certain metrics. For simplicity's sake our implementation chooses the $k$ initial point with equal probability.

\paragraph{}
We used a \textit{Map-Reduce} procedure to draw those those $k$ point

\section{Implementation}

\paragraph{Key}
This \textit{MapReduce} procedure doesn't use a key, so only one reducer will 
be spawned by the framework, in our \textit{Hadoop} implementation we used the 
\texttt{NullWritable} data-type as output key to reduce traffic.

\paragraph{Mapper}
The mapper assigns a random label to each data point, this value is not be confused with the \textit{MapReduce}'s key; finally it emits the tuple made of the label and the data point.

\paragraph{Combiner and Reducer}
Then, the combiner sorts the tuples by their label and emits the first $k$ 
smallest labels and their samples. Finally, the reducer performs the same 
operations as the combiner, emitting $k$ values.

\paragraph{Probability}
If the data are split equally among the nodes and the random numbers generator generates all numbers with equal probability, then all dataset's samples have circa equal probability to be drawn.

\paragraph{Complexity}
Let $n = |D|$ the number of samples, $N$ the number of nodes and $k$ the number of samples to draw; then the time complexity is

\[
T \in O \left(\dfrac{n}{N} \cdot \left(1 + \log(k)\right)\right)
\] 

and the space complexity is

\[
S \in \Theta \left(k\right)
\]

if a sorted data structure is used to store only the first $k$ smallest labels 
and their associated data.

\begin{algorithm}[H]
	\caption{Random select}\label{alg:random_map}
	\begin{algorithmic}
		\Require $k \in \mathbb{N}^+$
		\Procedure{Mapper}{nLine, $r$}
			\State $I \gets \Call{rand}{\;}$
				\Comment{Assign a random number to each file's row}
			\State \Return $\left< \texttt{null}, \left< I, r\right> \right>$
				\Comment{Constant key for all lines}
		\EndProcedure
		\vspace{.25cm}
		\Procedure{Combiner}{$S$}
			\State $L$ is a data structure \textbf{ordered} by key $I$
				\Comment{E.g. a binary search tree}
			\State $L \gets \emptyset$
			\While{$S \neq \emptyset$}
				\State $\left< \texttt{null}, \left< I, r\right> \right> \
					\gets \Call{read}{S}$
					\Comment{Read data from the mapper (or combiners)}
				%\State $L \gets L \bigcup \left\{ \left< I, r\right>\right\}$
				\State \Call{insert}{$L$, $\left< I, r\right>$}
				\If{$|L| > k$}
					\Comment{We store at most $k + 1$ elements}
					\State \Call{pop\_last}{L}
					\Comment{We remove the last element (sorted by $I$)}
				\EndIf
				
				\State \Call{next}{$S$}
			\EndWhile
			\ForAll{$ \left< I, r\right> \in L$}
				\State \Call{emitt}{$\left< \texttt{null}, \left< I, r\right> \right>$}
			\EndFor
		\EndProcedure
		\vspace{.25cm}
		\Procedure{Reducer}{$S$}
			\State \Call{Combiner}{$S$} \Comment{Same as the combiner}
		\EndProcedure
	\end{algorithmic}
\end{algorithm}

\chapter{K-means}



\chapter{Results}


\end{document}          
